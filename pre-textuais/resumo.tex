% -----------------------------------------------------------------------------
% Resumo
% -----------------------------------------------------------------------------

\begin{resumo}

\colorbox{yellow}{COPIADO TEMPLATE PFC II.}

O resumo deve ser apresentado em parágrafo único e justificado. Deverá conter no máximo 1800 caracteres contando-se os espaçamentos. O resumo deverá conter, obrigatoriamente: (1) Introdução, (2) Objetivos, (3) Amostra, (4) Procedimentos Experimentais; (5) Estatística (quando houver); (6) Resultados; (7) Discussão; e, (8) Conclusão. Todavia, não utilize tópicos colocando em negrito ou sublinhado cada um destes elementos mencionados acima. Utilize uma estrutura de texto de resumo em texto corrido que, por si só, seja capaz de caracterizar cada um dos elementos acima descritos como obrigatórios no resumo. Utilize linguagem objetiva. Nenhum dos tópicos do resumo deverá exceder em seu volume. A ‘Introdução’ deverá ser breve, assim como o ‘Objetivos’ (recomendo no máximo 2 linhas para cada tópico). Transcreva apenas as informações mais importantes nos métodos. Não é necessário detalhar tudo o que foi realizado nos ‘Procedimentos Experimentais’. A caracterização da ‘Amostra’ deve ser sucinta, apenas para dar uma noção de algumas de suas particularidades. Nos ‘Procedimentos Experimentais’, apenas relate as informações que consigam caracterizar os métodos utilizados. Nos ‘Resultados’, indique apenas os dados mais interessantes (que demonstraram significância estatística, por exemplo) e que foram discutidos posteriormente na ‘Discussão’. A ‘Discussão’ é a parte mais importante de um trabalho científico. Nela,, deverá ser apresentada a explicação para os resultados apresentados no estudo. Na ‘Discussão’ também serão verificadas as inferências do estudo sobre os resultados. Por fim, a ‘Conclusão’ deverá ser apresentada com possíveis implicações do estudo, generalizações dos resultados e sugestões de futuros estudos.

    \par\vspace{\baselineskip}

    \textbf{Palavras-chave}: De três a cinco palavras chaves que expressem o conteúdo e tema do trabalho. As palavras-chave devem ser separadas por ponto e vírgula (;).
\end{resumo}
