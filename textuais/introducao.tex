% -----------------------------------------------------------------------------
% Introdução
% -----------------------------------------------------------------------------

\chapter{INTRODUÇÃO} % Letra Maiúscula
\label{introducao}

\colorbox{yellow}{COPIADO TEMPLATE PFC II.}

Na introdução é exposto o tema central do trabalho e seu contexto em nossa sociedade e/ou curso, relacionando, caso exista, com o objeto foco do estudo (estudo de caso, caso seja este o método de pesquisa), seguido por: objetivo geral (com a definição do problema), objetivos específicos (itens secundários para se atender o objetivo geral).

Na introdução deve ser explicitado o problema de pesquisa de forma explícita, por exemplo com uma questão problema; ou de forma implícita, a partir da contextualização e dissertação acerca do tema. É comum o uso de citações da literatura para contextualizar o tema de pesquisa e o problema. É comum também conter uma justificativa sobre a importância do tema ser pesquisado, sempre considerando a literatura como referência para tal.

Para facilitar a padronização dos textos de PFC, são adotadas as seguintes regras de formatação do texto: Fonte: Times New Roman; Tamanho 12 para texto e 16 para títulos. Tamanho 10, para citações diretas longas e legendas (de Figuras e Tabelas); Espaço entre linhas e parágrafos: espaçamento entre linhas 1,5, e 0 pontos após o término de um parágrafo. Tamanho de folha A4, com margens superior e esquerda 3 cm, e direita e inferior 2 cm. Usar recuo no início de parágrafos. Os parágrafos devem estar justificados à direita e a esquerda. Usar como referência a formatação do presente texto.

A língua oficial para todos os trabalhos é a língua portuguesa. Caso o aluno opte pelo uso da língua inglesa para fins de publicação, o orientador deve consentir. Palavras escritas em idioma diferente da língua portuguesa devem estar em itálico (caso o português seja a língua oficial do trabalho). Todos os títulos de seções devem estar alinhados à esquerda e em negrito. Títulos das seções primárias devem ter todas as letras maiúsculas. Usar as dimensões de folha A4. Margens do texto: esquerda e superior: 3 cm. Direita e inferior: 2 cm.
Todas as páginas deverão ser numeradas, com exceção da capa e das páginas de anexo e apêndices (anexos e apêndices são opcionais). A numeração deve ser em algarismos arábicos, no canto inferior direito da folha.

O curso de Engenharia de Mecânica da UFSJ entende o PFC como sendo um dos passos finais para a obtenção do grau de bacharel em Engenharia Mecânica. Nesse sentido, o PFC visa avaliar a capacidade do graduando de desenvolver um trabalho acadêmico sob a orientação de um docente que atua no curso de Engenharia Mecânica em tema de interesse do discente.

\section{OBJETIVOS} % Letra Maiúscula

Na seção objetivo, são apresentados os objetivos geral e específicos para a obtenção dos resultados pretendidos. O objetivo geral é apresentado em uma única frase, de forma clara e concisa, além do resultado final pretendido ao término de seu artigo. Os objetivos específicos são objetivos secundários que devem ser realizados para alcançar o objetivo geral (principal), sendo estes elencados em tópicos, contendo um verbo no infinitivo, por tópico.

