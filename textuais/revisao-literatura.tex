% -----------------------------------------------------------------------------
% Revisão da Literatura
% -----------------------------------------------------------------------------

\chapter{REVISÃO DE LITERATURA} % Letra Maiúscula
\label{revisao}

Esta seção é a parte do trabalho onde o graduando apresentará o embasamento necessário para o desenvolvimento do trabalho. É o item que contém a maioria das citações, descrevendo o estado da arte do tema abordado. Sugere-se na maioria dos casos o uso de citações indiretas. As citações diretas devem ser usadas somente se as exatas palavras do autor forem extremamente importantes. Geralmente a revisão contém citações de métodos/abordagens propostas para resolver o problema em estudo; de objetos de estudo (problema) nos quais o método foi aplicado, de como foram medidos os resultados; e de conceitos e definições caso sejam importantes para o desenvolvimento da pesquisa. Equações também podem ser usadas para quantificar as definições.