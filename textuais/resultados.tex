% -----------------------------------------------------------------------------
% Resultados
% -----------------------------------------------------------------------------

\chapter{RESULTADOS} % Letra Maiúscula
\label{resultados}

Consiste na descrição detalhada dos resultados obtidos ou esperados com a execução da metodologia proposta. Estes resultados têm de ser mensuráveis com números, e de preferência com valores monetários (o quanto se espera de ganho ou redução de perda de recursos) ou dados estatísticos no caso de revisão da literatura.

São utilizados três elementos não textuais: Figuras, Tabelas e Equações. A indicação ou chamada para as figuras, tabelas e equações deve ser feita no texto, antes destes aparecerem e pela numeração. Os termos Figura, Tabela e Equação sempre são escritos com a inicial maiúscula quando fizerem referência a um elemento não textual presente no trabalho. Por convenção, são utilizadas sequências crescentes em algarismos arábicos para numerar Figuras, Tabelas e Equações, uma sequência para cada elemento não textual. Não use expressões como “Figura abaixo”, “Tabela acima”, “Equação a seguir” e similares.

Utilize figuras com boa resolução. Uma figura ilegível atrapalha mais do que ajuda. Tabelas nunca devem ser inseridas como Figuras. As legendas de figuras e tabelas são dispostas acima destas e centralizadas. As referências (fontes) das tabelas e figuras são apresentadas abaixo das mesmas, a exemplo da Figura 1 e Tabela 1. 